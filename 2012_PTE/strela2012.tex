\documentclass[aps,12pt,final,notitlepage,oneside,onecolumn,nobibnotes,
  nofootinbib,superscriptaddress,noshowpacs,centertags]{revtex4}

% from http://www.maik.ru/pub/tex/maik/maik.rty
\textheight = 235mm
\textwidth  = 165mm

\usepackage{amssymb,amsmath}
\usepackage{mathtext}

\usepackage[T2A]{fontenc}
\usepackage[utf8]{inputenc}
\usepackage[english,russian]{babel}

\usepackage{xspace}
\usepackage{graphicx}
\graphicspath{{figures/}}

\usepackage{caption2}
\setcaptionwidth{\linewidth}
\renewcommand{\captionlabeldelim}{.~}
\renewcommand{\captionlabelfont}{\bfseries}
\renewcommand{\captionfont}{\footnotesize}

\makeatletter
\renewcommand \thefigure{\@arabic\c@figure}
\renewcommand \thetable{\@arabic\c@table}

\newcommand{\np}{\ensuremath{np \rightarrow pn}\xspace}
\newcommand{\dpfrag}{\ensuremath{dp \rightarrow ppn}\xspace}
\newcommand{\dpret}{\ensuremath{dp \rightarrow (pn)p}\xspace}
\newcommand{\dpchex}{\ensuremath{dp \rightarrow (pp)n}\xspace}

\begin{document}

\selectlanguage{russian}

\title{Экспериментальная установка СТРЕЛА \\
  для изучения зарядово-обменных процессов}
\author{\firstname{В.~В.}~\surname{Глаголев}}
\affiliation{Объединенный институт ядерных исследований \\
  Россия, 141980, Дубна Московской обл., ул. Жолио-Кюри, 6}
\author{\firstname{Д.~А.}~\surname{Кириллов}}
\affiliation{Объединенный институт ядерных исследований \\
  Россия, 141980, Дубна Московской обл., ул. Жолио-Кюри, 6}
\author{\firstname{Г.}~\surname{Мартинска}}
\email{gabriela.martinska@upjs.sk}
\affiliation{Университет им. П.~Й.~Шафарика \\
  Словакия, 04154, Кошице, Шробарова 2}
\author{\firstname{Я.}~\surname{Мушински}}
\affiliation{Объединенный институт ядерных исследований \\
  Россия, 141980, Дубна Московской обл., ул. Жолио-Кюри, 6}
\affiliation{Институт экспериментальной физики САН \\
  Словакия, 04001, Кошице, Ватсонова 47}
\author{\firstname{Н.~М.}~\surname{Пискунов}}
\affiliation{Объединенный институт ядерных исследований \\
  Россия, 141980, Дубна Московской обл., ул. Жолио-Кюри, 6}
\author{\firstname{Й.}~\surname{Урбан}}
\affiliation{Университет им. П.~Й.~Шафарика \\
  Словакия, 04154, Кошице, Шробарова 2}

\begin{abstract}
  Описывается экспериментальная установка СТРЕЛА, предназначенная для
  исследования зарядово-обменных процессов в дейтрон-протонных столкновениях в
  области энергий выше 1~ГэВ. Установка представляет одноплечевой магнитный
  спектрометр, основными элементами которого являются дрейфовые камеры. Было
  проведено тестовое облучение в пучке дейтронов импульса 3.5~ГэВ/с на
  ускорительном комплексе Нуклотрона ЛФВЭ ОИЯИ. Разработаны и протестированы на
  реальных событиях алгоритмы для поиска и реконструкции треков в дрейфовых
  камерах. Полученное значение пространственного разрешения камер лежит в
  диапазоне 90--120~мкм, что позволяет осуществить изучение интересующего нас
  процесса.
\end{abstract}

\maketitle

\input{strela2012_main}

\bibliography{references}
%\bibliographystyle{maik}

\end{document}
