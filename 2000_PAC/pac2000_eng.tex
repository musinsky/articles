\documentclass[a4paper,12pt]{article}

\usepackage[english]{babel}
\usepackage{graphicx}
\graphicspath{{figures/}}

\setlength{\textheight}{25 cm}
\setlength{\textwidth}{16.5 cm}
\setlength{\oddsidemargin}{0.5 cm}
\setlength{\topmargin}{-2.2 cm}
\setlength{\footskip}{2\baselineskip}
\setlength{\headheight}{1\baselineskip}

\begin{document}

\section{Physics goal}

\hspace{0.5cm}
The task to build up the theory of the nucleon-nucleon scattering, specially
for the region above 1 GeV, remains still actual. To have a complete sample
of data what enables to carry out such informations as e.g. the phase analysis,
experiments with double and
triple scatterings are necessary. However, for investigation of the spin-flip
contribution into the np$\to$pn charge exchange, the deuteron interacting with
proton can be used, where the neutron is in a defined spin-orbital state.

The possibility to use the charge exchange reaction on the unpolarized
deuteron for determining the spin dependent part of the np$\to$pn charge
exchange has been proposed by A.B. Migdal~\cite{Mig} and
I.Y. Pomeranchuk~\cite{Pom}. The effect can be qualitatively understood in
the following way. The nucleons bound in the deuteron may be in $^{3}S_{1}$
and $^{3}D_{1}$ (T=0) spatial and spin symmetric states but their charge states
should be antisymmetric. In the charge exchange process a charge
symmetric state of two protons is produced and to fulfill the Pauli principle
at a conserved spatial symmetry, the spin flip ( $^{1}S_{0}$ or $^{1}D_{2}$
states of the two final protons )
of the scattered nucleon ensures the asymmetric total wave function. In this
way, the spin dependent part of the elementary charge exchange will be
reflected through the probability of the charge exchange on the deuteron.

\begin{center}
  \vspace*{-1cm}
  \unitlength=1.70mm
  \special{em:linewidth 0.7pt}
  \linethickness{0.7pt}
  \begin{picture}(81.33,92.67)
    \put(10,51){\vector(1,0){15}}
    \put(10,71){\vector(1,0){15}}
    \put(28,56){\circle{3.33}}
    \put(28,47){\circle*{3.33}}
    \put(31,56){\line(1,0){53}}
    \put(31,47){\line(1,0){53}}
    \put(28,71){\circle{3.33}}
    \put(31,71){\line(1,0){53}}
    \put(46,71){\circle*{3.33}}
    \put(46,56){\circle*{3.33}}
    \put(46,47){\circle*{3.33}}
    \put(84,71){\circle*{3.33}}
    \put(84,56){\circle*{3.33}}
    \put(84,47){\circle*{3.33}}
    \put(33,74){\vector(0,1){5}}
    \put(33,58){\vector(0,1){5}}
    \put(33,48){\vector(0,1){5}}
    \put(75,63){\vector(0,-1){5}}
    \put(75,48){\vector(0,1){5}}
    \put(80,73){\vector(0,1){5}}
    \put(75,80){\vector(0,-1){5}}
    \put(46,71){\line(4,5){6.5}}
    \put(46,56){\line(4,5){6.5}}
    \put(53.4,65){\circle{3.33}}
    \put(53.4,80){\circle{3.33}}
    \put(76,73){\line(2,5){3}}
    \put(17,73){\makebox(0,0)[cc]{$n$}}
    \put(46,68){\makebox(0,0)[cc]{$p_t$}}
    \put(46,43){\makebox(0,0)[cc]{$p_s$}}
    \put(84,68){\makebox(0,0)[cc]{$p^|$}}
    \put(84,44){\makebox(0,0)[cc]{$p_s$}}
    \put(84,53){\makebox(0,0)[cc]{$p^|$}}
    \put(57.4,66){\makebox(0,0)[cc]{$n^|$}}
    \put(57.4,81){\makebox(0,0)[cc]{$n^|$}}
    \put(17,53){\makebox(0,0)[cc]{$d$}}
    \put(4,51){\makebox(0,0)[cc]{b)}}
    \put(4,71){\makebox(0,0)[cc]{a)}}
  \end{picture}
\end{center}
\vspace{-6cm}
Fig. 1. Elementary np$\to$pn (a) and dp$to$(pp)n (b) charge exchange reactions.\\

In the figure 1 we show very schematycly images of two processes:
a) np$\to$pn and b)~dp$\to$(pp)n reactions, i.e. charge-exchange on the
simplest nucleus - deuteron. The vertical arrows present the directions
of the nucleon spins relatively arbitrary quantisation axes. Interactions
in the second process are going only if spin break-up for scattering
proton has place (Pauli princip). In the way the deuteron present here an
slip filter.

The mathematical formalism within the impuls approximation was developed
by Dean~\cite{Dea} and others. The general formula for the differential
cross section of the dp$\to$(pp)n looks like:
\begin{equation}
  \left( \frac{d\sigma }{dt}\right) (pd\rightarrow n(pp))=[1-F_d]\left(
  \frac{d\sigma _{nf}}{dt}\right) +[1-\frac{1}{3}F_d]
  \left(\frac{d\sigma _f}{dt}\right),
\end{equation}
where F$_d$ stands for the deuteron form factor, $\frac{d\sigma _{nf}}{dt}$ and
$\frac{d\sigma _f}{dt}$ are non-spin flip and spin flip parts, respectively,
of the np$\to$pn differential cross section. At zero transfers
$(\vert t \vert\sim 0)$, when F$_d$=1 the expression (1) simplifies to:
\begin{equation}
  \frac{d\sigma }{dt}(pd\rightarrow n(pp))=\frac 23\frac{d\sigma _f}
       {dt}(np\rightarrow pn).
\end{equation}
The last formula is applicable if at least two conditions are fulfilled:
\begin{enumerate}
\item the momentum transfer of the quasielastic np scattering is small,
\item the intrinsic momenta p of the nucleons in the deuteron are small.
\end{enumerate}

The second condition means S-wave dominance in the deuteron wave function.
It can be seen, e.g. in Fig 2, where the S wave probability distribution
is shown as a function of the nucleons intrinsic momenta. In the region
below p=0.07 GeV/c this probability practically does not depend on the nucleon
intrinsic momenta.

% ******* Figure 2 *********
\begin{figure}[hbt]
  \begin{center}
    \includegraphics[width=8cm]{wavedtr.pdf}
    %\mbox{\epsfig{figure=wavedtr.eps,width=8.cm}}
  \end{center}
  \vspace{0.4mm}
  Fig.2. The S wave probability as a function of the nucleon Fermi momentum. \\

\end{figure}


Both the above mentioned conditions can be fulfilled simultaneously, if one
selects events in the laboratory frame ( fast deuteron impings on the proton
target ) containing two fast protons  at small production angle, with momenta
close to the half of the deuteron's one. We would like to emphasize, that this
task can be realized successfully in the beams of accelerated deuterons only.
In the case of a deuteron target the two protons are too slow to be detected
and the reaction cannot be identified.

To answer the question of the spin flip contribution of the np$\to$pn process
as it follows from expression (2), it is
necesssary to turn to the experimental data on np$\to$pn charge exchange
differential cross section at t=0. Such data have been obtained at Brookhaven
\cite{Fri}
for the region of 1-8 GeV and could
be reasonably approximated by the function 1/p$^2$, where p is the momentum
of the incoming
neutron. The data imply ${d\sigma\over dt}$(0)=(36.9$\pm$ 3.0) $(GeV/c)^2$
at 1.67 GeV/c.  The task now is to compare the differential cross section
of the exchange on the deuteron, obtained in our experiment at t=0 with
those for the np$\to$pn at the corresponding beam energy.


\section{Bubble chamber experiment}


\hspace{0.5cm}
The first experimental data have been taken by the JINR 1m hydrogen bubble
chamber ~\cite{gla,Ala} in a full solid angle geometry, at a deuteron momentum of
3.35 GeV/c. The use of nuclear
beams impinging on a fixed proton target makes all the fragments of the
incoming nuclei fast in the laboratory frame and, thus, they can be detected,
well measured and identified practically without losses. On the other hand,
almost all the losses due to the chamber threshold momenta are concentrated
in the elastic channel. These conditions
allow to study the reactions containing not more than one
neutral particle in an exclusive approachs.

The  dp$\to$ppn reaction can be divided
into two channels:\\
\begin{enumerate}
\item the charge retention channel, where the proton is the fastest
  seconadary particle with respect to the deuteron rest frame and
\item the charge exchange channel, where the neutron is the fastest
  secondary particle with respect to the deuteron rest frame.
\end{enumerate}
The separation of these two channels is illustrated in Fig. 3, showing
the distribution of the four momentum transfer squared~t, between the impinging
proton and the secondary neutron. Quantity defined in such a way does not
depend on the final proton state, whether it is
a spectator or it participates in the reaction
(indifferent to the proton interferency). Hatched part in
Fig.3 - charge exchange interactions dp$\to$(pp)n.

% ******* Figure 1 *********
\begin{figure}[hbt]
  \begin{center}
    \includegraphics[width=8cm]{dist.pdf}
    %\mbox{\epsfig{figure=dist.eps,width=8.cm}}
  \end{center}
  \vspace{0.4mm}
  Fig.3. Distribution of four momentum transfer squared from the target proton
  to neutron for the dp$\to$ppn events. \\

\end{figure}

For a reasonable interpolation of $d\sigma/dt$ to t=0, it is inevitable
to select a region of production angles, excluding the high momentum
tails of the nucleon intrinsic motion and also the region of momentum
transfers, where other more complicated mechanisms than the quasielastic
scattering come into play. The differential cross sections
for four different values of the two selected protons production angles are
illustrated in Fig. 4.  The shape
of the distribution at small $\vert t \vert$ does not change with the increase
of the production angle while
the contribution of higher $\vert t \vert$ increases.

% ******* Figure 3 *********
\begin{figure}[hbt]
  \begin{center}
    \includegraphics[width=8cm]{tpp2345.pdf}
    %\mbox{\epsfig{figure=tpp2345.eps,width=8.cm}}
  \end{center}
  \vspace {0.4cm}
  Fig.4. Distributions of $\vert t \vert$ for the charge exchange channel with
  the production angles of both protons below 2,3,4 and 5 degrees. \\

\end {figure}

% ******* Figure 8 *********
\begin{figure}[hbt]
  \begin{center}
    \includegraphics[width=8cm]{sigmaok.pdf}
    %\mbox{\epsfig{figure=sigmaok.eps,width=8.cm}}
  \end{center}
  \vspace{0.4mm}
  Fig. 5. Differential cross section at t=0 when both protons are within
  a cone of opening angle $\Theta$. \\

\end{figure}

A typical value of the production angle is
$\Theta=arctg(p_f/p_0)=1.6^{\circ}$, where
the experimental and theoretical maxima ($p_f$=50 MeV/c) of the Fermi
momentum distribution of the nucleons in the deuteron state as a measure
of transverse momentum and the longitudinal momentum per nucleon
in the laboratory frame is $p_0$=1.67 GeV/c.
It leads to the result that the optimal value for the opening angle
of a cone within the two protons are produced is around 3$^{\circ}$.

Fig. 5 demonstrates the differential cross section charge exchange
reaction at t=0 as a function of the cut, imposed
on the proton production angles. It can be seen, that around $\theta$=3$^\circ$
the quantity ${d\sigma\over dt}(0)$ reaches the level of
${2\over3}{d\sigma\over dt}(0)$ of the elementary np$\to$pn process.
For the value of $\theta$=3$^\circ$ we get the contribution of the spin
flip part to the elementary np$\to$pn of $0.94 \pm 0.15$.

The obtained contribution of course depends on the unknown  systematical
errors of the
differential cross section of the elementary np$\to$pn charge exchange.
In any case the contribution obtained is large enough and does not exclude
the full spin flip in the np$\to$pn process. It means, that in the beams
of vector  polarized deuterons one can get an additional possibility
to use the studied reaction for more sophisticated purposes, e.g. to
explain the interference mechanism in the two protons system. The
spectator and the scattered protons are then tagged, because the first
one conserves the beam polarization and the latter (scattered) is then
polarized oppositely.


\section{The new STRELA setup}

\hspace{0.5cm}  The geometry, detectors and the background were estimated using:
1) Monte Carlo simulations of the   reaction,
2) real events from 17 different channels of dp interactions.
The following parameters were used as inputs to the model: the spot of the
deuteron beam at the target - 1 cm; the beam momentum - 3.5 GeV/c;
the thickness of the liquid hydrogen target - 10 cm;
the length of the magnet - 1m and the magnetic field fixing at 1 T;
the distances from the target to the magnet 1m and to the first detector 7 m;
the transverse size of the detector ~5 cm;
the  Gartenhaus-Moravcsik deuteron wave function  and the experimental
slopes of the Nd and NN elastic scattering.

These parameters can be optimized according to the geometrical requirements
of the setup. The experimental and Monte Carlo estimations of the background from the
one-proton events of the   reaction, i.e. from the charge retention deuteron
breakup, gave approximately the same results. It is expected not more
than 400 one-proton events per one two-proton event, when the aperture
of the event selection is ~0.5$^o$.
The effect-background ratio improves by increasing this angle.

In the case of one-proton events, the detected
protons are spectators and they can give a good basis for the cross section
normalization.

Thus, two protons, each with half of the deuteron beam momentum, must be
detected. This ensures that events at 0$^o$ with respect to the beam direction
with small relative momentum of the nucleon internal motion in the deuteron
and small transferred momentum to the scattered neutron are selected.
Therefore the minimal necessary conditions for the detection of the proposed
effect are possible to realize. On the first stage the experimental scheme,
shown in Fig. 6, is proposed.
% ******* Figure 9 *********
\begin{figure}[hbt]
  \begin{center}
    \includegraphics[width=9cm]{image2.pdf}
    %\mbox{\epsfig{figure=Image2.eps,width=8.cm}}
  \end{center}
  \vspace{0,4mm}
  Fig.6.  Layout of the experimental setup. \\

\end{figure}

The slow extraction deuteron beam is incident
on a 10-cm liquid hydrogen target LH2 and the analyzing magnet D separates
the primary deuterons and secondary particles. The ionization chamber IC
measures a flux of the deuteron beam. The scintillator monitors T1-T3
and M1-M2 control the intensity and position of the beam. The scintillator
counters S1-S4 with diameters of 48 mm determine the angular and momentum
acceptances (10$\%$) and trigger the event. The Cherenkov counters C1-C3 with
polystirol radiators of the size 50*50*16 mm$^3$ select the two-proton events.
The expected resolution about 10$\%$ allows to separate reliably the two
and one-proton events. The Cherenkov counter C4 with a quartz radiator
serves for pion suppression. The signal amplitudes from all counters are
recorded for all the trigger events. To make the event selection more
efficient, the possibility to include the signal amplitude from
one of the Cherenkov counters in the trigger logic is foreseen.

\section{Proposed data points, rates and first test of  STRELA setup}

\hspace{0.5cm}
With 5.10$^{-2}$ msr angular acceptance and a momentum acceptance of 5$\%$
of the setup the proposed data points (2000 two proton events per energy)
at the deuteron intensity of 10$^7$ per spill are given in the Table below.
Each data point requires tuning of the Nuclotron and the additional time to
check detectors and conditions of beam parameters, neither of them is included
in to the time shown in the last coloumn of the table.
\vspace*{5mm}

\begin{tabular}{|c|c|c|c|c|c|}    \hline
  N    &        Td (GeV)&       Deuteron & One proton & Two proton & Time \\
  & &        momentum  & event  & events  & (min) \\
  & &(GeV/c)&   &       &       \\ \hline
  1   & 1.41 &  2.7  &  3100 &  8  &    50 \\  \hline
  2   & 1.49 &  2.8  &  3100 &  8  &    50 \\  \hline
  3   & 1.58 &  2.9  &  3100 &  8  &    50 \\  \hline
  4   & 1.66 &  3.0  &  3200 &  9  &    50 \\  \hline
  5   & 1.75 &  3.1  &  3200 &  9  &    50 \\  \hline
  6   & 1.83 &  3.2  &  3300 &  10 &    40 \\  \hline
  7   & 1.92 &  3.3 &   3400 &  10 &    40 \\  \hline
  8   & 2.00 &  3.4 &   3600 &  10 &    40 \\  \hline
  9   & 2.09 &  3.5 &   3800 &  11 &    40 \\  \hline
  10  & 2.18 &  3.6 &   4200 &  11 &    40 \\  \hline
  11  & 2.27 &  3.7 &   4400 &  12 &    40 \\  \hline
  12  & 2.36 &  3.8 &   4800 &  13 &    30 \\  \hline
  13  & 2.45 &  3.9 &   5100 &  14 &    30 \\  \hline
  14  & 2.54 &  4.0 &   5400 &  15 &    30 \\  \hline
\end{tabular}
\vspace*{5mm}

In March,2000 the first run of STRELA setup on the extraction beam of NUCLOTRON
has passed. The key possibility of adjustment of a telescope on a maximal
stripping (Fig.7) and selections of two-proton events was shown (Fig.8).
% ******* Figure 10 *********
\begin{figure}[hbt]
  \begin{center}
    \includegraphics[width=8cm]{s3ok.pdf}
    %\mbox{\epsfig{figure=S3ok.eps,width=8.cm}}
  \end{center}
  \vspace{0,4mm}
  Fig.7. Experimental and Monte Carlo distributions of stripping protons.
\end{figure}
% ******* Figure 11 *********
\begin{figure}[hbt]
  \begin{center}
    \includegraphics[width=8cm]{strela.pdf}
    %\mbox{\epsfig{figure=strela.eps,width=8.cm,angle=0}}
  \end{center}
  \vspace{0,4mm}
  Fig.8. Cherenkov amplitude for one- and two-proton events.\\
\end{figure}

\begin{thebibliography}{99}
\bibitem{Mig}   A.B.Migdal ZhETF 28, 1955, p.3
\bibitem{Pom} I.Pomeranchuk DAN USSR, 1951, LXXVIII, N2, p.249
\bibitem{Dea} N.W.Dean,//Phys.Rev.D5,1972,p.461
\bibitem{Fri} J.L.Friedes et al.//Phys.Rev.Lett 15,1965,pp.38-41
\bibitem{gla} V.V.Glagolev, Nucl.Phys.B (Proc.Suppl.) 36 (1994) 509-512
\bibitem{Ala} B.S.Aladashvili et al. //Nucl.Phys.B86, 1975, p.461
\end{thebibliography}
\end{document}
