\documentstyle[12pt,epsfig]{article}
%\textwidth 155mm
\setlength{\textheight}{25 cm}
%\textheight 230mm
\setlength{\textwidth}{16.5 cm}
\setlength{\oddsidemargin}{0.5 cm}
\setlength{\topmargin}{-2.0 cm}
\setlength{\evensidemargin}{-0.5 cm}
\setlength{\footskip}{2\baselineskip}
\setlength{\headheight}{1\baselineskip}
\begin{document}

\begin{center}
{\Large\bf Spin-Dependent $np \rightarrow pn$ Amplitude Estimated
\hspace {125mm}
from  $dp\rightarrow ppn$
}
\end {center}
\bigskip
\begin{flushleft}
V.V. Glagolev$^{1}$, J. Hlav\'{a}\v{c}ov\'{a}$^{2}$,
M.S. Khvastunov$^{1}$,
N.B. Ladygina$^{1}$, G. Martinsk\'{a}$^{3}$,\\ J. Mu\v sinsk\'y$^{1,3}$,
B. Pastir\v c\'ak$^{4}$, N.M. Piskunov$^{1}$, T. Siemiarczuk$^{5}$
J. Urb\'{a}n$^{3}$ \\
\bigskip

\small{
$^{1}$Joint Institute for Nuclear Research,141980 Dubna,Russia\\
$^{2}$Technical University, Park Komensk\'{e}ho 2, SK-04200, Ko\v{s}ice, Slovak Republic\\
$^{3}$University of P.J.\v{S}af\'{a}rik, Jesenn\'{a} 5, SK-04154 Ko\v{s}ice, Slovak Republic \\
$^{4}$Institute of Experimental Physics SAS, Watsonova 47, SK-04353, Ko\v{s}ice, Slovak Republic \\
$^{5}$Institute of Nuclear Studies,ul. Hoza 69, Warsaw,  PL-00 681, Poland
\\}

\end{flushleft}

\bigskip

\begin{abstract}
\noindent
An estimation of the spin-dependent part of the $np\to pn$ charge exchange
amplitude was made on the basis of  $dp\to (pp)n$ data, taken
at 1.67 GeV/c per nucleon in a full solid angle arrangement.
The
$np\to pn$ amplitude turned out to be  entirely spin-dependent.
This result shows
new possibilities for experiments using polarized deuteron beams and
polarized proton targets.\\

PACS numbers: 25.10.+s; 25.45.-z; 25.45. Kk
\end{abstract}
\newpage

\section{INTRODUCTION}
\indent

The task to build up the theory of the nucleon-nucleon scattering, specially
for the region above 1 GeV, is an outstanding issue and therefore new
 experimental data are appreciated. A complete determination of spin-dependent
nucleon-nucleon elastic amplitudes requires many measurements. It requires
both spin correlation  parameters $A_{NN}$, etc. and spin transfer parameters
$K_{NN}$, etc.~\cite{Bil}.
 However, in some cases the picture may be simplified.
For example in the case of the investigation
of the spin-dependent
contribution to the elementary $np\to pn$ charge exchange reaction,
 unpolarized deuteron-proton
interactions can be used,
where the neutron spin-orbital state is well established.

The possibility to use the charge exchange reaction on the unpolarized
deuteron for determination of the spin-dependent part of the $np\to pn$ charge
exchange was emphasized partly in the series of works  ~\cite{Mig}-\cite{Bugg}.
%~\cite{Pom},~\cite{Lap},~\cite{Dean1},~\cite{Dean2},~\cite{Bugg}.
The effect can be understood qualitatively in
the following way. Two nucleons, bound in the deuteron may be in $^{3}S_{1}$
and $^{3}D_{1}$ ($T=0$) spatial and spin symmetric states; their isospin
is antisymmetric. In charge exchange at 0$^{\circ}$, the transition from
$^{3}S_{1}$ or $^{3}D_{1}$
to a charge
symmetric $^{1}S_{0}$ or $^{1}D_{2}$
state of two protons requires spin flip, in order
to satisfy the Pauli principle and ensure an anti-symmetric total
wave function.
 In this
way, the spin-dependent part of the elementary charge exchange amplitude
will be
reflected through the probability of the charge exchange process
on the deuteron.

In the general case the nucleon-nucleon amplitude in the centre of mass
can be presented as
\begin{equation}
M=a+b(\vec\sigma \hat n)(\vec\sigma _i \hat n)+
c[(\vec\sigma \hat n)+(\vec\sigma _i \hat n)]+
e(\vec\sigma \hat m)(\vec\sigma _i \hat m)+
f(\vec\sigma \hat l)(\vec\sigma _i \hat l),
\end{equation}
where the orthonormal basis
\begin{equation}
\hat l =\frac {\vec k +\vec k^\prime}{|\vec k +\vec k^\prime|},~~
\hat m =\frac {\vec k -\vec k^\prime}{|\vec k -\vec k^\prime|},~~
\hat n =\frac {\vec k \times\vec k^\prime}{|\vec k \times\vec k^\prime|}
\end{equation}
introduced in  ~\cite{Gold} is used. The vectors
$\vec k$ and $\vec k^\prime$ are
the initial and final momenta, respectively, $\vec\sigma $ and
$\vec\sigma_i$ are the Pauli matrices corresponding to
the fast particle and the struck nucleon from the deuteron, respectively.

In the impulse approximation the $dp$ - charge exchange
differential cross section at  small momentum transfer $|t|$ is related to the
$NN$-amplitudes via
\begin{equation}
\left( \frac{d\sigma }{dt}\right) (pd\rightarrow n(pp))=[1-F_d(t)]\left(
\frac{d\sigma _1}{dt}\right) +[1-\frac{1}{3}F_d(t)]
\left(\frac{d\sigma _2}{dt}\right),
\end{equation}
where
\begin{eqnarray}
\frac{d\sigma _1}{dt}=|a|^2+|c|^2 ~~~~~~~~
\frac{d\sigma _2}{dt}=|b|^2+|c|^2+|e|^2+|f|^2
\end{eqnarray}
 $F_d (t)$ denotes the deuteron form factor
 and the coefficients $a,~b,~c,~e$ and $f$ refer to spin invariants of the
 elementary charge exchange amplitude Eq. (1) \cite{Dean1},\cite{Ala}.

In this paper we consider the case, when the scattering angle $\theta$
is very small, close to zero. Under such kinematical conditions
one obtains
\begin{eqnarray}
b=e ~~~ \rm {and} ~~~c=0
\end{eqnarray}
and for the elementary cross section a simple expression can be written,
\begin{eqnarray}
\frac{d\sigma _1}{dt}=|a|^2 ~~~~~~~~
\frac{d\sigma _2}{dt}=2|b|^2+|f|^2 .
\end{eqnarray}
The amplitude $a$ is spin-independent, and $b$ and $f$ are spin-dependent.

At momentum transfer $\vert t \vert\sim 0$, when $F_d (0)$ = 1,
Eq. (3) reduces to:
\begin{equation}
\frac{d\sigma }{dt}(pd\rightarrow n(pp))=\frac 23\frac{d\sigma _2}{%
dt}(np\rightarrow pn).
\end{equation}
Thus, the $dp$-charge exchange differential cross section is fully determined
by spin-dependent parts of elementary $np\to pn$ amplitude.

During recent years  considerable progress has been achieved in solving
the problem of  construction of the scattering matrix.
In the intervening period a substantial amount of new $np$ data has been
accumulated.
Nowdays the $pp$ analysis is extended up to a laboratory kinetic energy of
2.5 GeV; the $np$ analysis was truncated at 1.3 GeV. While
the  situation in the complete reconstruction of the $pp$ scattering
 amplitude in the region above 1 GeV
 is  already satisfactory, the same cannot
be stated for $np$ scattering \cite{Arn}.

The aim of the present study was the extraction of information on
 the elementary process $np\to pn$. The existing data on that reaction
are still very scanty and concern mainly the $d\sigma/dt$ distribution.
As a consequence of the definite isotopic spin of the two protons, the
study of the $dp\to (pp)n$ reaction in the region of the
symmetric spatial part of their wave function
may give information on the spin structure of the
 elementary amplitude. A study of the deuteron-proton charge exchange
 differential cross section $d\sigma/dt$
allows one to estimate  the spin-dependent contribution to the $np\to pn$
reaction amplitude. Such experimental data do not yet exist.
Acceleration of deuterium beams allows to use them for the determination
of the spin-dependent part of elementary $np\to pn$ charge exchange process
in a wide region of energy.


\section{EXPERIMENT}
The experimental data were taken with the JINR 1-m hydrogen bubble
chamber in a full solid angle geometry and at a incident deuteron momentum of
3.35 GeV/c. The use of nuclear
beams impinging on a fixed proton target makes all the fragments of the
incoming nuclei fast in the laboratory frame, and thus they can be detected,
well measured and identified practically without losses. On the other hand,
almost all losses, due to the chamber threshold momenta, are concentrated
in the elastic channel. These conditions
allow one to study  reactions containing not more than one
neutral particle in an exclusive approach. A more detailed description
of the experimental set-up and the processing chain can be found
in ~\cite{hbc}.

The recorded pictures were scanned twice for all topologies. Measurements on three
projections were used for  geometrical reconstruction and  subsequent
kinematical analysis of events, if an event failed on any of the processing
steps, all four existing projections were remeasured. The geometrical
 reconstruction and the kinematical analysis were carried out using
an appropriate version of the CERN program package based on the HYDRA
library ~\cite{hyd}.
The ionization of  charged secondary particles were estimated visually.
The complete data summary tape contains 237413 events of $dp$ interactions.
A sample of 102757 events fitting the reaction $dp\to ppn$ was collected.
The studied events of the  $dp\to ppn$ reaction  could
be divided in a natural way into two channels:\\
\begin{enumerate}
\item the charge retention channel, where the proton is the fastest
 secondary particle with respect to the deuteron rest frame, and
\item the charge exchange channel, where the neutron is the fastest
secondary particle with respect to the deuteron rest frame.
\end{enumerate}
As a result 85239 events are attributed to the charge retention channel
and 17518 events corresponding to the charge exchange reaction.
The separation of these two channels is illustrated in Fig. 1, showing
the distribution of the four momentum transfer squared~$t$, between target
proton and secondary neutron in the laboratory frame. A quantity
defined this way does not
depend on the final proton state, whether it is a spectator or participant
(indifferent to the proton interferency).

The first attempt to determine the contribution of the spin-dependent
amplitude of the $np\to pn$ elementary charge exchange from the differential
cross section of the $dp\to (pp)n$ reaction \cite{Ala}, using the available
$pp$ and $np$ scattering data, was carried out in the intial stage of the
experiment on a relatively small part of the processed events. The $pp$ and
$np$ scattering  data
were ambiguous, the statistics poor and correspondingly the obtained
estimate was indefinite, though it dropped a hint to the enhanced role
of the spin-dependent amplitude in the $np\to pn$ charge exchange.

Using the final statistics of over $10^{5}$ events of the $dp\to ppn$ reaction
we came back to this problem for two reasons:
\begin{enumerate}
\item To make a direct estimate of the $dp\to (pp)n$ differential cross section
at $t=0$ on the basis of the Dean formalism \cite{Dean1}.
\item To estimate the possibilities and limitations of a prepared counter
experiment \cite{Baz}.
\end{enumerate}

In the study of high-energy nuclear reactions, the  coordinate system
 in which the nucleus is at rest is customarily used. For that reason all
the physical quantities given below are in the deuteron rest frame if
not stated otherwise.

\section{RESULTS AND DISCUSSION}

In order to extract the spin-dependent part of the $np\to pn$ amplitude from the
$dp\to (pp)n$ charge exchange data applying Eq. (7), at least the two
following conditions have to be satisfied:
\begin{enumerate}
\item the momentum transfer of the quasielastic $np$ scattering is small,
\item the intrinsic momenta $(q)$ of the nucleons in the deuteron are small.
\end{enumerate}

The second condition means $S$-wave dominance in the deuteron wave function,
which is shown
in Fig. 2, where the $S$ wave probability distribution
is plotted as a function of the nucleon intrinsic momentum. In the region
below $q$ = 0.07 GeV/c this probability practically does not depend on the nucleon
intrinsic momentum.

Both the above mentioned conditions can be fulfilled simultaneously, if one
selects events in the laboratory frame containing two fast protons  at small
production angle relative to the incoming deuteron momentum and with momenta
 close
to half that of the deuteron. We would like to stress, that this task
can be realized successfuly using accelerated deuteron beams.
In the case of a deuteron target the two protons are too slow to be detected
and the reaction cannot be identified. All dedicated
charge exchange experiments on a deuteron so far have been carried out
 with proton beams.

In addition to the above mentioned, as follows from Eq. (2), to address
the question of the spin-dependent contribution to the $np\to pn$ amplitude,
one has to turn to the data on the differential cross section at $t=0$.
Such kind of data do exist \cite {Fri}, \cite {She}.
They have been obtained at Brookhaven \cite {Fri} for the region
of 1-8 GeV and can
be reasonably approximated by $1/p^2$, where $p$ is the momentum of the incoming
neutron. These data imply at $t=0$\\
 $d\sigma/dt$ = (36.9 $\pm$ 3.0) mb/(GeV/c)$^2$
at 1.67 GeV/c.  The quoted value is in good agreement with that obtained from
the $np\to pn$ experimental data \cite{She} at 1.729 GeV/c, fitted by a sum of
 two
 exponentials $d\sigma/dt$ = (36.5 $\pm$ 1.4) mb/(GeV/c)$^2$ (without
systematical uncertainties). Similar behaviour of the differential cross
section $d\sigma/dt$ at $t=0$ has been observed below our energies,
a peak is present at $u\to 0$ ~\cite{Bon}.

The task is to compare the differential cross section
of the charge exchange on the deuteron, obtained in our experiment at $t=0$ with
that for the $np\to pn$ at the corresponding beam energy.

For a reasonable approximation of $d\sigma/dt$ to $t=0$, it is inevitable
to select a region of production angles, excluding the high momentum
tails of the nucleon intrinsic motion and also those regions of momentum
transfer, where more complicated mechanisms than  quasi-elastic
scattering can come into play. The changes of the differential cross section
 at four different values of the two proton selection angles are
illustrated in Fig. 3. With the increase of the angle the character
of the $d\sigma/dt$  distribution at small $\vert t \vert$ remains unchanged,
while at  larger $\vert t \vert$
the contribution increases.

An estimation of the production angle $\theta$ can be obtained, using
the experimental and theoretical maxima ($p_f$ = 50 MeV/c) of the Fermi
 momentum distribution of the nucleons in the deuteron as a measure
of transverse momentum and the value of $p_0$ = 1.67 GeV/c for
 the longitudinal momentum per nucleon
in the laboratory frame. It provides the
value
$\theta=arctg(p_f/p_0)=1.6^{\circ}$. This requires that the two protons
should be produced within a cone having an opening angle of
$\approx$ 3$^{\circ}$.

In this contest we remind the used definitions: the slowest proton
in the deuteron rest frame
is referred to as a spectator; the other one we call scattered. If
no cuts are imposed on the proton production angles their momentum
distributions differ significantly  as  shown in Fig. 4.

 The spectator
proton distribution (full line) follows a curve, typical for the Fermi
momentum distribution in the deuteron. When the above
mentioned cut of 3$^\circ$ is applied to the production angles
of spectator and scattered
proton, their momentum distributions overlap, as one can see
 in Fig. 5.

The differential cross section for small values of $\vert t\vert$ is
displayed in Fig. 6 together with the curves, corresponding to a fit
of $d\sigma/dt$ = $ae^{bt}$ to the data.
 The fit gives the
following results: $a$ = (19.0 $\pm$ 1.1) mb/(GeV/c)$^2$
($\chi^2$/$ND$ = 4.3/8) for the interval of
 $\vert t\vert$ = $\lbrace 0.0-0.02\rbrace$(GeV/c)$^2$ (Fig. 6a)
 and  $a$ = (23.1${^+3.6 \atop _ -3.1}$) mb/(GeV/c)$^2$
($\chi^2$/$ND$ = 1.0/2)
for the interval
 $\vert t\vert$ = $\lbrace 0.0-0.004 \rbrace$(GeV/c)$^2$ (Fig. 6b).

 Fig. 7 demonstrates the differential cross section at $t=0$ as a
 function of the cut, imposed
on the proton production angles. It can be seen in agreement with
Eq. (7), that around
$\theta$ = 3$^\circ$
the quantity $d\sigma/dt$ at $t=0$ reaches the level of
2/3 ($d\sigma/dt$) at $t=0$ of the elementary $np\to pn$ process.
For the value of $\theta$ = 3$^\circ$ we get the contribution of the
spin-dependent part to the elementary $np\to pn$ of $0.94  \pm  0.15$. The obtained
contribution of course depends on the systematical errors
of the elementary $np\to pn$ charge exchange cross section $\approx$
20\% \cite{She}.
In any case the obtained probability is large enough and does not exclude
the amplitude being 100\% spin-dependent in $np\to pn$.
The estimate of the spin-independent part of the
amplitude in ~\cite{Ala} based on a different approach and poor statistics
(the number of charge exchange events is more than one order less),
did not exclude the nonzero spin-dependent contribution. Therefore
the present results obtained here in a more straightforward way using
Eq. (7) are not in contradiction to our earlier findings ~\cite{Ala}.
Our experiment allows us to extract only the spin-dependent part of the
$np\to pn$ charge exchange amplitude. Therefore the obtained results
cannot be directly compared with the data taken in polarized proton
beam experiments, e.g.~\cite{Rans}. In a future study of the
 process $dp\to (pp)n$ using
a beam of polarized deuterons one could
separate the two spin-dependent terms in the amplitude of the charge exchange
reaction $np\to pn$, one of which does not conserve while  the other
conserves the projection of the nucleon spin onto the direction of
momentum at the transition of the neutron into the proton ~\cite{Gla}.
The proposed method is applicable in the energy range up to 10 GeV, where
the charge exchange cross section is not too small. At these energies
the phase-shift analysis due to large number of partial waves is complicated.
%\newpage
\section{CONCLUSION}
\indent
The study of the $dp\to (pp)n$ reaction in full solid angle conditions
showed, that the amplitude of the elementary $np\to pn$ charge exchange
at 1.67 GeV/c is practically  fully spin-dependent.
The obtained result offers new possibilities to measure
the energy dependence of this effect by simultaneous use of both  polarized
deuteron beams and polarized proton target.
\noindent
\section{ACKNOWLEDGMENTS}
\indent
This work was in part supported by the Grant Agency for Science at
the Ministry of Education of the Slovak Republic (grant No. 1/8041/01).


\begin{thebibliography}{99}
\bibitem{Bil} S.M. Bilenkij, L.I. Lapidus, R.M. Ryndin,
 Usp. Fiz. Nauk (in Russian)\\{\bf LXXXIV}, 243 (1964)
\bibitem{Mig} A.B. Migdal, J. Exp. Theor. Phys. (in Russian) {\bf 28}, 3 (1955)
\bibitem{Pom} I. Pomeranchuk, Dokl. Akad. Nauk (in Russian)  {\bf LXXVIII}, 249 (1951)
\bibitem{Lap} L.I. Lapidus, J. Exp. Theor. Phys.(in Russian) {\bf 32},1437 (1957)
\bibitem{Dean1} N.W. Dean, Phys. Rev. {\bf D5}, 1661 (1972)
\bibitem{Dean2} N.W. Dean, Phys. Rev. {\bf D5}, 2832 (1972)
\bibitem{Bugg} D.V. Bugg, C. Wilkin, Nucl. Phys. {\bf A167}, 575 (1987)
\bibitem{Gold} M. Goldberger, K. Watson, Collision Theory, New York,
Wiley, 1966
\bibitem{Ala} B.S. Aladashvili et al., Nucl. Phys. {\bf B86}, 461 (1975)
\bibitem{Arn} R.A. Arndt, I.I. Strakovsky, and R.L. Workman,
Phys. Rev. {\bf C50}, 2731\\ (1994);
 R.A. Arndt, I.I. Strakovsky, and R.L. Workman, Phys. Rev. {\bf C56}, 3005\\
 (1997);
 A.de Lesquen et al.,
Eur. Phys.J. {\bf C11}, 69 (1999); C. Lechanoine-Leluc,\\ F. Lehar, P. Winternitz,
and J. Bystricky, J.Physique {\bf 48},
 985 (1987); R.A. Arndt,\\ I.I. Strakovsky, and R.L. Workman, Phys. Rev.
 {\bf C62}, 034005 (2000)
\bibitem{hbc} B.S. Aladashvili et al., Nucl. Instrum. Methods, {\bf 129},
109 (1975); B.S. Aladashvili et al., JINR Commun. 1-7645,Dubna,1973
\bibitem{hyd} V. Framery et al., Hydra Topical Manual TQ Title Package,
CERN Program\\ Library (1982)
\bibitem{Baz} S.N. Bazylev et al., in Proc. of the Int.Workshop
"Relativistic nuclear physics\\ from hundreds MeV to TeV", Slovak Republic,
Stara Lesna, June 26-July 1, \\2000, p.234
\bibitem{Fri} J.L. Friedes, H. Palevsky, R.L. Stearns, and R.J.Sutter,
 Phys. Rev. Lett. {\bf 15}, 38 (1965)
\bibitem{She} P.F. Shepard, T.J. Devlin, R.E. Mischke, and J. Solomon,
 Phys. Rev. {\bf D10}, 2735 (1974)
\bibitem{Bon} B.E. Bonner et al., Phys. Rev. Lett. {\bf 41}, 1200 (1978)
\bibitem{Rans} R.D. Ransome et al., Phys. Rev. Lett. {\bf 48}, 781 (1982)
\bibitem{Gla} V.V. Glagolev, V.L. Lyuboshitz, V.V. Lyuboshitz, N.M. Piskunov,
JINR Commun.\\ E1-99-280, Dubna, 1999
\end{thebibliography}
\newpage
\section{FIGURE CAPTION}
\noindent
Fig. 1. Distribution of four momentum transfer squared from the target proton
 to the neutron for the $dp\to ppn$ events (black
area - charge exchange events, white area - charge retention events). \\

\noindent
Fig. 2. The $S$ wave probability as a function of the nucleon Fermi momentum.
Full line - Paris, dashed line - Bonn A, dotted line - Bonn B, dash - dotted
line - Bonn C wave function.\\

\noindent
Fig. 3. Distribution of $\vert t \vert$ for the charge exchange channel
 with  the
production angles of both protons below 2, 3, 4 and 5 degrees. \\

\noindent
Fig. 4. Momentum distributions of  spectator ( full line) and
scattered protons (dashed line) in the deuteron rest frame. \\

\noindent
Fig. 5. Momentum distributions of spectator ( full line) and
scattered protons (dashed line) in the deuteron rest frame. The proton
 production angles
in the laboratory frame are below 3$^{\circ}$. \\


\noindent
Fig. 6. Differential cross section of the charge exchange reaction in
the region of small $\vert t \vert$. The production angle of both protons
in the laboratory frame is in the interval 0-3$^{\circ}$. (b) shows
finer binning than (a). \\

\noindent
Fig. 7. Differential cross section at $t=0$ when both protons are within
a cone of opening angle $\theta$. The full line corresponds to 2/3 ($d\sigma/dt)_
{np\to pn}$ at $t$ = 0,  the dashed lines shows the uncertainties. \\

\newpage
\section{FIGURE}
%                 xxxxx  FIGURE 1 xxxxx
\begin{figure}[hbt]
\begin{center}
\mbox{\epsfig{figure=fig1.eps,width=8cm}}
\end{center}
\vspace{0.4 mm}
Fig.1. Distribution of four momentum transfer squared from the target
proton to the neutron for the $dp\to ppn$ events (black area - charge exchange
channel, white area - charge retention channel).
\end{figure}
%                        xxx FIGURE 2 xxxxx
\begin{figure}[hbt]
\begin{center}
\mbox{\epsfig{figure=fig2.eps,width=8.cm}}
\end{center}
\vspace{0.4mm}
Fig.2. The $S$ wave probability as a function of the nucleon Fermi momentum.
Full line - Paris, dashed line - Bonn A, dotted line - Bonn B, dash - dotted line-
Bonn C wave function.
\end{figure}
%                xxxxx FIGURE 3 XXXXXXXX
\begin{figure}[hbt]
\begin{center}
\mbox{\epsfig{figure=fig3.eps,width=8.cm}}
\end{center}
\vspace{0.4mm}
Fig.3. Distribution of $\vert t \vert$ for the charge exchange channel with
the production angles of both protons below 2, 3, 4 and 5 degrees.
\end{figure}
%           xxxxxx   FIGURE 4   xxxxxx
\begin{figure}[hbt]
\begin{center}
\mbox{\epsfig{figure=fig4.eps,width=8cm}}
\end{center}
\vspace{0.4mm}
Fig.4. Momentum distributions of  spectator (full line) and
 scattered protons (dashed line) in the deuteron rest frame.
\end{figure}
%                xxxxxxx FIGURE 5 xxxxxxx
\begin{figure}[hbt]
\begin{center}
\mbox{\epsfig{figure=fig5.eps, width=8cm}}
\end{center}
\vspace{0.4mm}
Fig.5. Momentum distributions of  spectator (full line) and  scattered
protons (dashed line) in the deuteron rest frame. The proton production angles
 in the laboratory frame are below 3$^{\circ}$.
\end{figure}
%                 xxxxxxx FIGURE 6 xxxxxxxx
\begin{figure}[hbt]
\begin{center}
\mbox{\epsfig{figure=fig6.eps,width=8.cm}}
\end{center}
\vspace{0.4mm}
Fig.6. Differential cross section of the change exchange reaction in the
region of small $\vert t \vert$. The production angle of both protons in
the laboratory frame is in the interval 0-3$^{\circ}$. (b) shows finer
binning than (a).
\end{figure}
%              xxxxxxx     FIGURE 7 xxxxxxxxx
\begin{figure}[hbt]
\begin{center}
\mbox{\epsfig{figure=fig7.eps,width=8.cm}}
\end{center}
\vspace{0.4mm}
Fig.7. Differential cross section at $t=0$ when both protons are within
a cone of opening angle $\theta$. The full line corresponds to 2/3
$(d\sigma/dt)_{np\to pn}$ at $t=0$, the dashed lines shows the uncertainties.
\end{figure}




\end{document}


